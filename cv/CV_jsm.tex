\documentclass[11pt,a4paper,sans]{moderncv}  
\moderncvstyle{casual}
\usepackage[scale=0.8]{geometry}
%\usepackage[colorlinks=true, linkcolor=Red]{hyperref}
%\usepackage[utf8]{inputenc}
%\usepackage[T1]{fontenc}
%\usepackage[latin1]{inputenc}
%\usepackage[french]{babel}
%\usepackage{hyperref}
%\renewcommand*{\namefont}{\fontsize{38}{40}\mdseries\upshape}
\renewcommand*{\namefont}{\fontsize{28}{30}\mdseries\upshape}

%\moderncvcolor{blue}
\definecolor{lightblue}{rgb}{0.32, 0.68, 0.9}
\definecolor{webpageblue}{rgb}{0.32, 0.68, 0.8}
\definecolor{cincblue}{rgb}{0.067, 0.553, 0.8}
%\definecolor{darkred}{rgb}{0.71,0.18,0.04}
%\definecolor{darkred}{rgb}{0.66,0.13,0.09}
%\definecolor{darkred}{rgb}{0.48,0.14,0.04}
%\definecolor{darkred}{rgb}{0.6,0.3,0.7}

\colorlet{color1}{cincblue}

% personal data
\name{Dr.}{Joana Soldado Magraner}
\title{Curriculum Vitae}
\address{Carnegie Mellon University}{Pittsburgh}{US}    
%\phone[mobile]{+1~4126269186}
%\phone[fixed]{+2~(345)~678~901}
%\phone[fax]{+3~(456)~789~012}
\email{jsoldadomagraner@cmu.edu}
%\homepage{www.johndoe.com}        
%\social[linkedin]{john.doe}                   
%\social[twitter]{jdoe}                             
\social[github]{jsoldadomagraner}                       
%\extrainfo{additional information}         
\photo[64pt][0.4pt]{jsm_picture}
%\quote{"Be ashamed to die until you have won some victory for humanity" Horace Mann} 
%"All models are wrong, but some are useful"
% to show numerical labels in the bibliography (default is to show no labels); only useful if you make citations in your resume
%\makeatletter
%\renewcommand*{\bibliographyitemlabel}{\@biblabel{\arabic{enumiv}}}
%\makeatother
%\renewcommand*{\bibliographyitemlabel}{[\arabic{enumiv}]}% CONSIDER REPLACING THE ABOVE BY THIS
% bibliography with mutiple entries
%\usepackage{multibib}
%\newcites{book,misc}{{Books},{Others}}



\begin{document}

\makecvtitle


\section{Education}

\cventry{2013--2018}{PhD}{}{The Gatsby Computational Neuroscience Unit, University College London (UCL)}{London, UK}{PhD program in Theoretical Neuroscience and Machine Learning}  % arguments 3 to 6 can be left empty

\cventry{2011--2013}{MSc}{}{Institute of Neuroinformatics, ETH-UZH}{Zürich, Switzerland}{Master's program in Neural Systems and Computation} 

\cventry{2009--2010}{BSc\&MSc}{}{RWTH-Aachen}{Aachen, Germany}{Erasmus programme exchange year, BSc+MSc in Physics.} 

\cventry{2006--2011}{BSc\&MSc}{}{Universitat de València}{València, Spain}{Licenciatura (BSc+MSc) in Physics.} 


\section{Academic Employment History}

\cventry{2019-present}{Postdoctoral Research Associate}{}{Carnegie Mellon University}{Pittsburgh, US}{}  % arguments 3 to 6 can be left empty

\cventry{2018-2019}{Postdoctoral Research Associate}{}{The Gatsby Computational Neuroscience Unit, UCL}{London, UK}{}

%\cventry{2013--2018}{PhD Researcher}{}{The Gatsby Computational Neuroscience Unit, UCL}{London, UK}{} 

\cventry{2012--2013}{Research Assistant}{}{}{}{HIFO, Brain Research Institute, University of Zürich}

\section{Postdoc project}
\cvitem{project}{\emph{Probing PFC dynamics and computation with patterned microstimulation perturbations.}}
\cvitem{advisors}{Byron Yu and Matthew Smith}

\section{PhD thesis}
\cvitem{thesis}{\emph{Linear dynamics of evidence integration in contextual decision making.}}
\cvitem{supervisor}{Maneesh Sahani}
\cvitem{minor}{\emph{First-order approximation of cross-validation for automatic regularization of estimators}}
\cvitem{supervisor}{Aapo Hyvarinen}
%\cvitem{description}{Short thesis abstract}

\section{Master thesis}
\cvitem{thesis}{\emph{Integration of evidence in Recurrent Neural Networks with synaptic normalization.}}
\cvitem{supervisors}{Valerio Mante, Michael Pfeiffer and Kevan Martin}
%\cvitem{description}{Short thesis abstract}




\section{Additional research experience}
\subsection{Research projects}

\cventry{2013}{Msc short project}{Learning Reward States in a Probabilistic Categorisation Task}{}{}{Institute of Neuroinformatics, ETH-UZH Zürich. \newline{}
%Research short project II, Master's programme in Neural Systems and Computation. \\
Supervisor: Michael Pfeiffer.}

\cventry{2012}{Msc short project}{Analysing two photon microscopy data from recordings of long-range projection neurons in somatosensory cortex of awake behaving mice}{}{}{HIFO, Brain Research Institute, University of Zürich. \newline{}
%Research short project I, Master's programme in Neural Systems and Computation.\\
Supervisors: Jerry Chen and Fritjof Helmchen.\\} 


\subsection{Research fellowships}

\cventry{2009}{JAE-Intro (CSIC Research Introduction Scholarship)}{}{ATLAS Silicon Forward Tracker Group and GRID Computing Group}{IFIC, CSIC-UV Particle Physics Institute, València, Spain}{}{}

\cventry{2008}{Research internship}{}{Environmental Radioactivity Laboratory}{UV, Universitat de València, Spain}{}


%Fellowships
%Studienfinanzierung
%Irlanda


\section{Academic experience}

\subsection{Mentoring}

\cventry{2021-present}{Advisor and collaborator}{Yuki Minai}{PhD Thesis}{PhD program in Neural Computation and Machine Learning, CMU}{'A closed-loop electrical microstimulation framework to control neural activity and behavior'}

\cventry{2021-2023}{Supervisor}{Lucas Nadolskis}{MSc Thesis}{Biomedical Engineering, CMU}{'Exploring top-down visual pathways using micro-stimulation and its applications to cortical visual prosthesis'}
%\\ Blind student who successfully completed a master research program adapted to his disability.}

\cventry{2021 summer}{Mentor}{}{Neuromatch Academy}{}{}

\cventry{2020-2021}{Supervisor}{Mathew Hall}{MSc Thesis}{Biomedical Engineering, CMU}{‘A convolutional neural network for generalized and efficient spike classification’}

\cventry{2017-2018}{Supervisor}{Eugenie Ordonneau}{BSc Natural Sciences Literature Review module, UCL}{}{‘Decision-making cortical circuits for motion perception in the saccadic system of primates’}

\subsection{Teaching}
%\cventry{2025}{Guest lecturer}{42-630 Intro to Neural Engineering}{Department of Biomedical Engineering, CMU}{graduate course.}{}

\cventry{2024-2025}{Guest lecturer}{18-698 Neural Signal Processing}{Department of Electrical and Computer Engineering, CMU}{graduate course.}{}

\cventry{2024-2025}{Future Faculty Program trainee}{Eberly Center, CMU}{}{}{A teaching training program with seminars, teaching observations with feedback (during guest lectures), a course design project and a statement of teaching philosophy project.}

\cventry{2023-2025}{Teaching Coordinator}{TReND-CaMinA school in Computational Neuroscience and Machine Learning Basics}{summer school}{}{Teaching and Research in Natural Sciences for Development in Africa
(TReND)}

\cventry{2023-2025}{Instructor}{TReND course in Computational Neuroscience and Machine Learning Basics}{summer school}{}{Machine Learning module: Dimensionality reduction techniques for neural data analysis}

\cventry{2016}{Teaching Assistant}{Society for Neuronscience (SfN)}{short course}{}{Data Science and Data Skills for Neuroscientists}

\cventry{2014}{Teaching Assistant}{}{Theoretical Neuroscience}{The Gatsby Unit, UCL}{PhD programme in Theoretical Neuroscience and Machine Learning}

\subsection{Reviewing}
\cventry{2022-2024}{Cosyne}{Reviewer}{Computational and Systems Neuroscience conference}{}{}
\cventry{2023}{Cell}{Co-reviewer}{Scientific journal}{}{}
\cventry{2023}{Communications Biology}{Co-reviewer}{Scientific journal}{}{}
\cventry{2021}{Nature}{Co-reviewer}{Scientific journal}{}{}
\cventry{2020}{Neuron}{Co-reviewer}{Scientific journal}{}{}
\cventry{2018}{NEURIPS}{Reviewer}{Neural Information Processing Systems conference}{}{}

\subsection{Conferences, workshops and schools}
\cventry{2023-2025}{Co-organiser}{TReND-CaMinA}{}{Summer school in Computational Neuroscience and Machine Learning Basics}{An intensive two-week course to teach African students the basics of Computational Neuroscience: 
a thriving and cost-effective research field to boost scientific capacity in the continent}
\cventry{2019}{Co-organiser}{CapoCaccia}{Cognitive Neuromorphic Engineering Workshop}{}{Working group: sRNNs stability, training and dynamics analysis}
\cventry{2019}{Co-organiser}{Cosyne}{Computational and Systems Neuroscience workshop}{}{Data, dynamics and computation: using data-driven methods to ground mechanistic theory}

\subsection{Boards and Commitees}
\cventry{2025}{Consultant}{}{REI-RICORS}{Redes de Investigación Cooperativa Orientadas a Resultados en Salud, Inflamacion y Neuroinflamacion}{Data science and statistics consultant}
\cventry{2020--2024}{Member}{}{IEEE Neuroethics working group}{}{Contributing to write guidelines for the use of neurotechnologies and discussing their ethical, legal, social, and cultural implications.}
\cventry{2012--2013}{Board Member}{}{Frei Denken Zürich}{}{Founded by an interdisciplinary group of students from Neuroscience, Medicine, Engineering, Philosophy and Ethics to promote ’Free Thinking’ and
rationality among students and the public.}
\cventry{2008--2009}{Student representative}{}{Physics Faculty Committee}{Universitat de València}{}
\cventry{2008--2009}{Board member}{}{Physics Student Association}{Universitat de València}{}



\section{Competitions and awards}
    
\cventry{2019}{NEUROTECH fellowship}{CapoCaccia}{Cognitive Neuromorphic Engineering Workshop}{}{} 
    
\cventry{2015}{Honourable mention}{IWSP7 poster prizes}{}{}{The international workshop on seizure prediction.\newline{}
\emph{Performance of synchrony and spectral-based features in early seizure detection: exploring feature combinations and effect of latency.}}

\cventry{2014}{Top ten ranking}{UPenn-Mayo Clinic Seizure Detection Challenge}{}{}{Kaggle Data Science contest for early seizure detection in epilepsy.\newline{}
A method employing synchrony and spectral-based features with a random forest classifier for early seizure detection. Ranked 9th out of 205 participants.}

%\cventry{2008}{First award}{ESPOU, Experimental Science Congress}{Pablo de Olavide University}{Sevilla, Spain}{\emph{Study of Radon-222 indoor concentration depending on environmental conditions}.\newline{}
%Research project conducted at the Environmental Radioactivity Laboratory, Universitat de València.}


\section{Congresses, workshops and symposia attended}

\cventry{2024}{CPPC}{Computational Properties of Prefrontal Cortex}{}{}{}
\cventry{2014--2023}{COSYNE}{Computational and Systems Neuroscience conference}{}{}{}
\cventry{2016,2022,2024}{SfN}{Society for Neuroscience meeting}{San Diego, USA}{}{}
\cventry{2022}{Bernstein Conference}{Bernstein Network in Computational Neuroscience}{Berlin, Germany}{}{}
\cventry{2019}{CapoCaccia}{Cognitive Neuromorphic Engineering Workshop}{}{}{}
\cventry{2015,2017}{NCCD}{Neural Coding, Computation and Dynamics workshop}{}{}{}
\cventry{2017}{TENSS}{Transylvanian Experimental Neuroscience Summer School}{Cluj-Napoca, Romania}{}{}
\cventry{2015}{IWSP7}{The international workshop on seizure prediction}{Melbourne, Australia}{}{}
\cventry{2012}{FENS-IBRO-Hertie Winter School: Brain Dynamics and Dynamics of Brain Diseases}{}{Austria}{}{}
%\cventry{2012--2013}{Swiss Computational Neuroscience Seminar Series}{}{ETH-UZH, EPFL, Uni Bern}{Switzerland}{}
%\cventry{2011}{Computational Astrophysics and Cosmology}{}{Universitat de Valencia}{Valencia, Spain}{}
%\cventry{2008}{ESPOU}{Experimental Science Congress}{Pablo de Olavide University}{Sevilla, Spain}{}{}


\section{Public engagement}
\cventry{2023-2024}{TReND}{Teaching and Research in Natural Sciences for Development in
Africa}{Outreach activities at local universities in Accra, Ghana and Kigali, Rwanda}{}{}
\cventry{2022}{SEMF Summer School}{Society for Multidisciplinary and Fundamental Research}{Multidisciplinary talks and courses for young researchers and the general public}{Universitat Politecnica de Valencia}{Invited talk}
\cventry{2019}{William Perkin High School STEM enrichment day}{Science workshop}{Sainsbury Wellcome Center Public Engagement Network}{London}{}
\cventry{2015-2017}{Science week}{}{Physics and Neuroscience talks}{Spanish high school Cañada Blanch, London}{}
\cventry{2013}{Robots on Tour}{}{ETH exhibitor assistant}{Artificial Intelligence Lab, Zürich}{}


\section{Publications}

\subsection{Journal Articles}

\cventry{2025}{Robustness of working memory to prefrontal cortex microstimulation}{}{JNeuroscience special issue: Computational Properties of the Prefrontal Cortex, \emph{(under review)}, preprint in \emph{bioRxiv}}{invited article.}{\underline{Joana Soldado-Magraner}, Yuki Minai, Matthew Smith and Byron Yu}

\cventry{2025}{Brain-computer interfaces as a causal probe for scientific inquiry}{}{\emph{Trends in Cognitive Sciences (under review)}}{invited review}{Asma Motiwala*, \underline{Joana Soldado-Magraner}*, Aaron Batista, Matthew Smith and Byron Yu}

\cventry{2024}{Inferring context-dependent computations through linear approximations of prefrontal cortex dynamics}{}{\emph{Science Advances}}{}{\underline{Joana Soldado-Magraner}, Valerio Mante and Maneesh Sahani}

\cventry{2024}{Examining funders’ roles in responsible research and innovation of medical neurotechnology}{}{\emph{Journal of Responsible Innovation}}{}{Denis Larrivee, Jennifer French, Alberto Antonietti, Zach McKinney, Noeline W Prins, \underline{Joana Soldado-Magraner}, Michael J. Young, and Laura Y. Cabrera}

\cventry{2024}{Applying the IEEE Neuroethics Framework to Intra-cortical Brain Computer Interfaces}{}{\emph{Journal of Neural Engineering}}{}{\underline{Joana Soldado-Magraner}*, Alberto Antonietti*, Jennifer French, Nathan Higgins, Michael J. Young, Denis Larrivee and Rebecca Monteleone}

\cventry{2018}{Brittleness in model selection analysis of single neuron firing rates}{}{\emph{PNAS (under 2nd revisions)}}{preprint in \emph{bioRxiv}}{Chandramouli Chandrasekaran, \underline{Joana Soldado-Magraner}, Diogo Peixoto, William T Newsome, Maneesh Sahani and Krishna V Shenoy}
% \href{doi: https://doi.org/10.1101/430710}{link}

\cventry{2013}{Behaviour-dependent recruitment of long-range projection neurons in somatosensory cortex}{Nature}{499, 336-340}{}{Jerry L. Chen, Stefano Carta, \underline{Joana Soldado-Magraner}, Bernard L. Schneider and Fritjof Helmchen}

\subsection{Conference Papers}

\cventry{2024}{MiSO: Optimizing brain stimulation to create neural activity states}{Neurips}{}{}{Yuki Minai, \underline{Joana Soldado-Magraner}, Matthew Smith and Byron Yu}

\cventry{2022}{Reexamining the ethical, legal, social, and cultural implications for cochlear implants through a novel neuroethics framework}{IEEE ISTAS 2022 proceedings}{}{}{Noeline Prins*, Rebecca Monteleone*, \underline{Joana Soldado-Magraner}, Joanne Nash, Michael J. Young and Laura Y. Cabrera.}


\section{Presentations}

\subsection{Invited talks}

\cventry{2024}{Robustness of prefrontal cortex networks under patterned microstimulation perturbations}{SfN}{Nanosymposium "Mechanisms of Working Memory and Cognitive Control in Prefrontal Circuits"}{}{Joana Soldado-Magraner}

\cventry{2024}{Robustness of prefrontal cortex networks under patterned microstimulation perturbations}{8th Computational Properties of Prefrontal Cortex Workshop}{Session "What can neural dynamics teach us about prefrontal function?"}{}{Joana Soldado-Magraner}

\cventry{2022}{Inter-areal patterned microstimulation selectively drives PFC activity and behavior in a memory task}{Bernstein conference}{Workshop "Distributed computations across brain regions"}{}{Joana Soldado-Magraner}

\cventry{2022}{High-order computations by neural population dynamics in the prefrontal cortex}{BARCCSYN}{}{}{Joana Soldado-Magraner}

\cventry{2021}{Context-dependent computations through linear dynamics in prefrontal cortex circuits.}{Janelia-HHMI Research Campus}{Computation and Theory Lecture Series}{}{Joana Soldado-Magraner}

\cventry{2019}{Linear dynamics of contextual decision-making}{CapoCaccia}{Session "Biological foundations of signal integration"}{}{Joana Soldado-Magraner}

\cventry{2019}{Inferring and interpreting neural dynamics during contextual decision making}{Cosyne}{Workshop "Data, dynamics and computation: using data-driven methods to ground mechanistic theory"}{}{Joana Soldado-Magraner}

\cventry{2018}{Linear dynamics of evidence integration in contextual decision making}{Oxford}{Neurotheory Forum (ONTF)}{}{Joana Soldado-Magraner}

\cventry{2016}{Do decision-related firing rates of dorsal premotor cortex neurons ramp or step on single trials?}{SfN}{Nanosymposium "Visual Decision Making"}{}{Chandramouli Chandrasekaran, Joana Soldado-Magraner, Diogo Peixoto, Maneesh Sahani and Krishna V. Shenoy}


\subsection{Poster presentations}

\cventry{2023}{Robustness of PFC networks under inter- and intra-hemispheric patterned microstimulation perturbations}{Cosyne}{selected poster}{}{Joana Soldado-Magraner, Yuki Minai, Matthew Smith and Byron Yu.}

\cventry{2022}{Inter-areal patterned microstimulation selectively drives PFC population activity across behavioral tasks}{SfN}{accepted poster}{}{Joana Soldado-Magraner, Yuki Minai, William Bishop, Matthew Smith and Byron Yu.}

\cventry{2022}{Inter-areal patterned microstimulation selectively drives PFC activity and behavior in a memory task}{Cosyne}{selected poster}{}{Joana Soldado-Magraner, Yuki Minai, William Bishop, Matthew Smith and Byron Yu.}

\cventry{2017}{Dynamically constrained vs unconstrained linear models of evidence integration in a contextual DM task}{NCCD}{selected poster}{}{Joana Soldado-Magraner, Valerio Mante and Maneesh Sahani}

\cventry{2015}{Linear dynamics of evidence integration in a contextual decision making task}{NCCD}{selected poster}{}{Joana Soldado-Magraner, Valerio Mante and Maneesh Sahani}

\cventry{2015}{Linear dynamics of evidence integration in a contextual decision making task}{Cosyne}{selected poster}{}{Joana Soldado-Magraner, Valerio Mante and Maneesh Sahani}

\cventry{2015}{Performance of synchrony and spectral-based features in early seizure detection: exploring feature combinations and effect of latency}{IWSP7}{invited poster}{}{Vincent Adam, Joana Soldado-Magraner, Wittawat Jitkrittum, Heiko Strathmann, Balaji Lakshminarayanan, Alessandro Davide Ialongo, Gergo Bohner, Ben Dongsung Huh, Lea Goetz, Shaun Dowling, Iulian Vlad Serban and Matthieu Louis}

\section{Online resources}

\subsection{Open-source code and teaching materials}
\cventry{2023-2024}{TReND-CaMinA course in computational neuroscience and machine learning basics}{}{Python notebooks, lecture slides and datasets}{freely available at the TReND course \href{http://www.github.com/trendinafrica/Comp_Neuro-ML_course}{\underline{Github repository}}}{TReND-CaMinA course teaching team (Coordinator: Joana Soldado-Magraner).}

\subsection{Methods reports}
\cventry{2015}{Seizure Detection Challenge The Fitzgerald team solution}{}{}{}{Vincent Adam, Joana Soldado-Magraner, Wittawat Jitkrittum, Heiko Strathmann, Balaji Lakshminarayanan, Alessandro Davide Ialongo, Gergo Bohner, Ben Dongsung Huh, Lea Goetz, Shaun Dowling, Iulian Vlad Serban and Matthieu Louis}

%\newpage


\section{Computer skills}
\cvitem{Coding}{MATLAB (advanced), Python (advanced), C++, R, Labview, NEST, Mathematica, Root}
\cvitem{OS}{Linux (Ubuntu), Mac OS X, Microsoft Windows}
\cvitem{Typesetting}{\LaTeX}
\cvitem{Version Control}{Github, svn}
\cvitem{Cluster Computing}{SLURM}



\section{Languages}
\cvitemwithcomment{Catalan}{Mother tongue}{}
\cvitemwithcomment{Spanish}{Mother tongue}{}
\cvitemwithcomment{English}{Proficiency}{}
%TOEFL iBT: 96 points, 2010. CAE, level C1, 2009
\cvitemwithcomment{German}{Intermediate}{DSH (Deutsche Sprachprüfung für den Hochschulzugang) level C1, 2010}
\cvitemwithcomment{Portuguese}{Conversational}{}


%\section{Non-academic work experience}
%\cventry{2012}{Cook}{}{Bar Milchbar}{}{Zürich, Switzerland}
%\cventry{2011--2012}{Waitress, cook}{}{Cafe Be\&So}{}{Zürich, Switzerland}
%\cventry{2008--2009}{Waitress}{}{Celtic Pub Max Max}{}{València, Spain}
%\cventry{2003--2011}{Meat preparations and delivery, office work}{}{Disricaem S.L. meat industry}{}{València, Spain}


%\section{Other memberships}
%\section{Additional interests and skills}
%\cvitem{Hobbies}{Reading about science, philosophy and history, learning languages, traveling, cooking, worldwide gastronomy, cultural exchange, mountain and sea sports (climbing, hiking, skiing, snorkelling and surfing).}
%\cvitem{}{Effective Altruism London board member, 2013-2016}
%\cvitem{}{Giving What We Can Switzerland board member, 2012-2013}
%\cvitem{}{Eager to work in groups and in highly multidisciplinary environments.}
%\cvitem{}{With a huge innate curiosity and always willing to learn.}



%\section{Extra 1}
%\cvlistitem{Item 1}
%\cvlistitem{Item 2}
%\cvlistitem{Item 3. This item is particularly long and therefore normally spans over several lines. Did you notice the indentation when the line wraps?}


%\section{References}
%\begin{cvcolumns}
 % \cvcolumn{Category 1}{\begin{itemize}\item Person 1\item Person 2\item Person 3\end{itemize}}
  %\cvcolumn{Category 2}{Amongst others:\begin{itemize}\item Person 1, and\item Person 2\end{itemize}(more upon request)}
%  \cvcolumn[0.5]{All the rest \& some more}{\textit{That} person, and \textbf{those} also (all available upon request).}
%\end{cvcolumns}

% Publications from a BibTeX file without multibib
%  for numerical labels: \renewcommand{\bibliographyitemlabel}{\@biblabel{\arabic{enumiv}}}% CONSIDER MERGING WITH PREAMBLE PART
%  to redefine the heading string ("Publications"): \renewcommand{\refname}{Articles}
%\nocite{*}
%\bibliographystyle{plain}
%\bibliography{publications}                       % 'publications' is the name of a BibTeX file



% Publications from a BibTeX file using the multibib package









%\bibliographystyle{plain}
%\bibliographystyle{apacite}
%\bibliography{soldado-magraner}                   % 'publications' is the name of a BibTeX file
%\nocitebook{book1,book2}
%\bibliographystylebook{plain}
%\bibliographybook{soldado-magraner}                   % 'publications' is the name of a BibTeX file
%\nocitemisc{misc1,misc2,misc3}
%\bibliographystylemisc{plain}
%\bibliographymisc{soldado-magraner}                   % 'publications' is the name of a BibTeX file

%\clearpage\end{CJK*}                              % if you are typesetting your resume in Chinese using CJK; the \clearpage is required for fancyhdr to work correctly with CJK, though it kills the page numbering by making \lastpage undefined


\end{document}


%% end of file `template.tex'.
